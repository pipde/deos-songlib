\documentclass{article}  
\input{header}

\htmltitle{songlib: reading notes}

\title{Songlib: reading notes\\
\date{Revision Date: \today}}

\author{Song Li Buser}

\begin{document}

\maketitle

\W\subsubsection*{\xlink{Printable Version}{readingSamples.pdf}}
\W\htmlrule


\section{Reading a set of notes}

In {\bf\it songlib}, one can read in a single note or
a set of notes with a single function call.
To
read in a set of notes, a call to {\it readScale} is
made. The prototype of {\it readScale} is:

\begin{verbatim}
    int readScale(char *directory,char *prefix);
\end{verbatim}

This reads in all the notes in the given directory with the given
prefix. The notes are read as RRA objects and then stored in
an array. The C note of ocatve 0 is stored at array slot 0,
The B note of octave 2 is stored at array slot 25, and so on.
For example, the following code reads in all the notes (as RRA
files) in the "/home/songlib/samples/guitar/" directory that begin with
the prefix "pick\_".

\begin{verbatim}
    #define DIR "/usr/local/share/samples/guitar/"
    #define PREFIX "pick_"

    int guitar = readScale(DIR,PICK);

    play(4.0,guitar,C,3);
\end{verbatim}

The function returns the instrument number assigned to notes that were
specified. The first call to {\it readScale} returns 0, the second returns 1,
and so on. The {\it readScale} command first looks for notes of the form:

\begin{verbatim}
    pick_c0.rra
    pick_c#0.rra
    pick_d0.rra
    ...
\end{verbatim}

If no notes are found, it then looks for notes of the form:

\begin{verbatim}
    pick_0.rra
    pick_1.rra
    pick_2.rra
    ...
\end{verbatim}

Notes of this form are assumed to correspond to midi note numbers.
If the note number is $x$, then it is placed at octave $x / 12$ and note
$x \% 12$.

Note that at least one note need be present in the directory.
If a request is made to play notes other than those found,
songlib will resample the closest found note to make the
desired note. However, the further away the found note,
the more artificial the resampling will sound. A good
rule of thumb is not to resample more than three or four
semitones away. Also, resampling a note to a higher pitch
will shorten the length of the resulting note.

\section{Reading a single note}

One can also read in a single note and then use the
note playing functions that take an RRA object as
an argument. To read in an RRA note, use the
{\it readRRA} function:

\begin{verbatim}
    RRA *readRRA(FILE *in,void (*handler)(RRA *,char *,void *))
\end{verbatim}

The first argument is a file pointer to the file containing the
desired note. The second note is a handler for non-standard tags.
For most songlib work, you do not need to supply a handler, so
a null value (zero) is passed as the second argument. Here
is an example:

\begin{verbatim}
    char *name = "stick.rra";
    RRA *note;
    FILE *fp;

    fp = readRRA(name,"r");
    if (fp == 0) Fatal("could not read file %s\n",name);
    note = readRRA(fp,0);
    fclose(fp);

    rplay(4.0,note);
\end{verbatim}

Notes read and played in this manner are not automatically
filtered. They can be manually filtered by calling the
\seed{builtInFilters}{built-in filters} appropriately.

For other note playing functions, see \see{notePlaying}.

\end{document}
