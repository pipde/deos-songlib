\documentclass{article}  
\input{header}

\htmltitle{songlib: note playing}

\title{Songlib: note playing\\
\date{Revision Date: \today}}

\author{John C. Lusth}

\begin{document}

\maketitle

\W\subsubsection*{\xlink{Printable Version}{notePlaying.pdf}}
\W\htmlrule

The major note playing functions/families are:

\begin{itemize}
\item
    \xlink{play}{play.html}
\item
    \xlink{chord}{chord.html}
\item
    \xlink{bend}{bend.html}
\item
    \xlink{trill}{trill.html}
\item
    \xlink{draw/resolve}{draw.html}
\item
    \xlink{silence}{silence.html}
\end{itemize}

Most functions have alternative forms. For example, the main note playing
function is {\it play}, but there are the following alternative forms:
{\it nplay}, {\it rplay}, and {\it dplay}.

The 'n' version of a function means the note is specified by a numbered
note rather than an octave/pitch pair. For example, these calls are
all equivalent:

\begin{verbatim}
play(Hd,guitar,3,C);
nplay(Hd,guitar,C3);
play(Hd,guitar,C3 / NOTES,C3 % NOTES);
nplay(HD,guitar,3 * NOTES + C);
\end{verbatim}

The 'r' version of a function indicates that an RRA object is passed
rather than an octave/pitch specification. These three calls are
equivalent:

\begin{verbatim}
play(Hd,guitar,3,C);
rplay(Hd,getNote(guitar,3,C));
rplay(Hd,getNumberedNote(guitar,C3));
\end{verbatim}

The 'd' version of a function indicates that raw amplitude data is passed
as an integer array and a length. The following calls to play and dplay
are equivalent:

\begin{verbatim}
RRA *r = getNote(guitar,3,C);
play(Hd,guitar,3,C);
dplay(Hd,r->data[0],r->samples);
\end{verbatim}

See also: \textcolor{red}{\xlink{noteNotations}{noteNotations.html}},\textcolor{red}{\xlink{rra}{rra.html}}

\end{document}
