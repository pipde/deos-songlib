\documentclass{article}  
\input{header}

\htmltitle{songlib: note notations}

\title{Songlib: note notations\\
\date{Revision Date: \today}}

\author{written by: Song Li Buser}

\begin{document}

\maketitle

\W\subsubsection*{\xlink{Printable Version}{template.pdf}}
\W\htmlrule

Songlib nominally uses a Western music scale with 12 notes (or semitones)
per octave. The following note constants are defined:

\begin{verbatim}
    #define C     0
    #define Cs    1
    #define Db    1
    #define D     2
    #define Ds    3
    #define Eb    3
    #define E     4
    #define F     5
    #define Fs    6
    #define Gb    6
    #define G     7
    #define Gs    8
    #define Ab    8
    #define A     9
    #define As   10
    #define Bb   10
    #define B    11
\end{verbatim}

Here, the ''s'' appelation denotes a sharp and ''b'' a flat.
These constants are used in conjuction with an octave specification
to specify which notes \see{play}, \see{chord}, and others
are to produce.

There are also ~~absolute~~ constants defined as well:

\begin{verbatim}
    #define C0    0
    #define Cs0   1
    #define Db0   2
    #define D0    2
    ...
    #define B0   11
    #define C1   12
    #define Cs1  13
    #define Db1  13
    #define D1   14
    ...
    ...
    ...
    #define B10  131
\end{verbatim}

These constants are
used with the
\seed{nplay}{notePlaying} and \seed{getNumberedNote}{filters}
commands,
among others.

\section{Percussion}

A special set of note constants is used for playing
percussion. By songlib covention, the following notes
correspond to the following percussion instruments:


\begin{tabular}{cll}
~~note suffix~~ & named constant~~~ & explanation\\
3 & BLOCK\_LOW & wood block, low\\
6 & BLOCK\_MIDDLE & wood block, middle\\
9 & BLOCK\_HIGH & wood block, high\\
12 & BASS\_LOW & bass drum, low\\
15 & BASS\_MIDDLE & bass drum, middle\\
18 & BASS\_HIGH & bass drum, high\\
21 & CLAP & hand clap\\
24 & COWBELL & cowbell\\
27 & CRASH & crash cymbal\\
30 & CRASH\_ALT & crash cymbal, alternate\\
33 & HAT\_CLOSED & high hat, closed\\
36 & HAT\_OPEN & high hat, open\\
39 & HAT\_PEDAL & high hat, pedal\\
42 & KICK & pedal drum\\
45 & KICK\_ALT & pedal drum, alternate\\
48 & RIDE\_ & ride cymbal\\
51 & RIDE\_ALT & ride cymbal, alternate\\
54 & RIDE\_BELL & ride cymbal, bell\\
57 & SNARE & snare drum\\
60 & SNARE\_ALT & snare drum, alternate\\
63 & STICK & stick on rim\\
66 & STICKS & stick on stick\\
69 & TIM\_LOW & timbale, low\\
72 & TIM\_MIDDLE & timbale, middle\\
75 & TIM\_HIGH & timbale, high\\
78 & TOM\_LOW & tomtom, low\\
81 & TOM\_MIDDLE & tomtom, middle\\
84 & TOM\_HIGH & tomtom, high\\
87 & CRASH\_CHOKE & short crash\\
90 & BELL & bell\\
93 & CHINA & china cymbal\\
96 & SPLASH & splash cymbal\\
99 & TAMBOURINE & tambourine\\
102 & SHAKER & shaker\\
105 & RIMSHOT & rim shot\\
108 & PERC8 & percussion\\
111 & PERC9 & percussion\\
114 & PERC10 & percussion\\
117 & PERC11 & percussion\\
120 & PERC12 & percussion\\
123 & PERC13 & percussion\\
126 & PERC14 & percussion\\
129 & PERC15 & percussion\\
\end{tabular}

Assuming the notes have been set up properly, one would
use these constants thusly:

\begin{verbatim}
    drumset = readScale("/usr/local/share/samples/drums/","hydrogen_");

    snplay(Q,drumset,COWBELL);
    snplay(I,drumset,CRASH);
\end{verbatim}

Drums are usually played with the {\it snplay} playing function, since
the entire note will play but the song will advance only the
requested number of beats. There is a version of {\it snplay} that
randomly introduces some slop into the amplitude, pitch and
timing of the note:

\begin{verbatim}
    for (i = 0; i < 4; ++i)
	drum(I,drumset,TOM_MIDDLE)
\end{verbatim}

This makes a repetitive series of beats sound less mechanical.
You can set the slop with the functions:

\begin{verbatim}
    void setSlopTiming(double);
    void setSlopAmplitude(double);
    void setSlopPitch(double);
\end{verbatim}

\end{document}
