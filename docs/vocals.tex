\documentclass{article}  
\input{header}

\htmltitle{Songlib: processing vocals}

\title{Songlib: processing filters\\
\date{Revision Date: \today}}

\author{written by: Song Li Buser}


\begin{document}

\maketitle

\W\subsubsection*{\xlink{Printable Version}{vocals.pdf}}
\W\htmlrule

\section*{Processing vocals}

This page gives a recipe for producing a rich sounding vocal track using
\songlib\ and {\it Audacity}.

\section*{Record multiple takes}

Using whatever recording software you like, record multiple takes of
your vocals. One way is to use the {\it record} program that comes with
\Songlib. Instructions can be found near the bottom of the \Songlib\ 
homepage. Another is Audacity.

However you record, save each take as a wave file. We will assume those
takes are named:

\begin{verbatim}
    take1.wav take2.wav take3.wav ...
\end{verbatim}

\section*{Cleaning up the dead air}

Often, there is a slight amount of noise in your recordings even when
you are not singing. Use the {\it rrasilence} filter to clean up your takes.
For example, to clean up {\it take1.wav}, you would run the command:

\begin{verbatim}
    wav2rra take1.wav | rrasilence | rra2wav > clean1.wav
\end{verbatim}

The {\it rrasilence} filter forces quiet sections of your vocals down to
absolute zero volume. To make the filter more or less aggressive,
see its options:

\begin{verbatim}
    silence -h
\end{verbatim}

\section*{Compressing your vocals}

Open up Audacity and import your mastered instrument track (you will
need to convert it WAV first) as an audio track.
In addition, import two of your clean WAV files and save your project.
Select all the tracks and the select the Compressor from the Effects
menu. Run the compressor to make the loudest parts of your vocals
no so loud.

\section*{Picking the best phrases}

Using the Time Shift Tool (double-headed horizontal arrow on
the tool bar), align the two tracks with the instrumental track.

Using the mouse, highlight the first phrase of your two vocal tracks.
The keyboard combination \verb!<Shift>-<spacebar>! will start playing the
selected vocals in a loop. You should also hear your instrumentals, as well.
Again, using the mouse, mute one of the tracks;
the {\it Mute} button is located to the
left of the track. Listen to the phrase that is not muted and then mute it
and unmute the other track. Listen to the newly unmuted phrase. Keep going
back and forth until you decide which version you like better. 
Press the spacebar to stop the loop. Highlight
the phrase in the track you don't like and enter \verb!<Ctl>-L!!
This will reduce the unwanted phrase to silence. If you did it correctly,
the phrase you prefered will remain.

Repeat this process for every phrase in the tracks.

Once you are done, highlight a region in both tracks and
enter \verb!<Ctl>-<Shift>-M! to merge the two tracks into a new, single track.
Save your project and then delete the two individual tracks.

Import your next clean take into Audacity and repeat the entire process
again until you have the best phrases from all your takes.

\section*{Adjust the timings of your phrases}

Play the entire song, stopping when you hear some vocals that are 
not quite lined up with the instrumentals. If the vocals are coming
in too late, higlight a small bit of the quiet region before the phrase
and enter \verb!<Ctl>-X! to remove the section.
Click on a region of quiet space
immediately after the phrase and enter \verb!<Ctl>-V! to paste in the deleted
section. This procedure will move the phrase earlier in time, leaving
subsequent phrases at their same positions.
For phrases that come in too early, cut time after the phrase
and paste the deleted portion before the phrase.

Repeat the entire process until all phrases are properly aligned.

Once you are happy with the timing,
duplicate your vocal track twice with
\verb!<Ctl-D>!. The two duplicate tracks will eventually become
your backing vocals.

\section*{Equalizing your vocals}

Run the equalizer from the Effects menu on your vocals. See the web for the
best curve for vocals.

\section*{Add backing tracks}

Select one of the duplicate tracks by double-clicking and then choose
{\it Change Pitch} on the {\it Effects} menu. 
Move the pitch up by a little bit (0.1 or 0.2 semitones). Do the same
for the other track, only pitch it down (-0.1 or -0.2 semitones).
At this point, the vocals should sound fuller and richer.

If the original vocals are on the sharp side, then pitch both tracks down,
one more than the other. Likewise, if the original vocals are flattish, pitch
both tracks up.

You can also time shift the duplicate tacks forward in time a little bit,
one more than the other to add a chorus effect.

Reduce the volume of the duplicate tracks gradually until any flanging
type artifacts cannot be heard.

\end{document}
