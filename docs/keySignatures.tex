\documentclass{article}  
\input{header}

\htmltitle{Songlib: key signatures}

\title{Songlib: key signatures\\
\date{Revision Date: \today}}

\author{written by: Song Li Buser}

\begin{document}

\maketitle

\W\subsubsection*{\xlink{Printable Version}{keySignatures.pdf}}
\W\htmlrule

\section*{Using key signatures}

The \songlib\ system allows you to compose 
within the structure of a key. By setting
a key within your \songlib\ program, you 
can use special chord generating functions
that do all the work for you in figuring
out what sounds pleasing (or not so pleasing
if you choose). To start, set the key 
by using the {\it setKey} function. Here are
some example calls to {\it setKey}:

\begin{verbatim}
    setKey(C);
    setKey(D);
    setKey(Gb);
\end{verbatim}

The default key for \songlib\ is C.

The next step is to set the mode using
the {\it setMode} function. The \songlib\ system
currently supports 9 modes:

\begin{itemize}
\item
    IONIAN
\item
    DORIAN
\item
    PHRYGIAN
\item
    LYDIAN
\item
    MIXOLIDIAN
\item
    AEOLIAN
\item
    LOCRIAN
\item
    MELODIC\_MINOR
\item
    HARMONIC\_MINOR
\end{itemize}

Here is an example call to {\it setMode}:

\begin{verbatim}
    setMode(AEOLIAN);
\end{verbatim}

The default mode for \songlib\ is {\sc IONIAN}.
The mode determines the `mood' of your composition,
with IONIAN, LYDIAN, and MIXOLIDIAN generating
a brighter mood and the rest generating darker moods.

Finally, call the chord function to generate
chords appropriate for your chosen mode. Here is
an example call to {\it c}, which plays the first
of seven chords in the chosen mode:

\begin{verbatim}
    c(1,beats,instrument,octave);
\end{verbatim}

The \verb!c(1,...)! function call will play a chord whose root note (or
base note) is C if the key is C, D if the key is D
and so on. The \verb!c(2...)! function call will play a chord whose
root note is one note higher, and so on.
To play individual notes, rather than chords, do something
similar to the following:

\begin{verbatim}
    play(beats,instrument,octave,degree(1));
    n(1,beats,instrument,octave,0);
\end{verbatim}

Both these calls will play a C note in the key of C, a D note
in the key of D and so on. Calling:

\begin{verbatim}
    play(beats,instrument,octave,degree(2));
    n(2,beats,instrument,octave,0);
    n(1,beats,instrument,octave,1);
\end{verbatim}

will all play the next higher note in the key, and so on.

If you following these conventions, you can change
your key and your mode just by calling {\it setKey} and
{\it setMode}, with no other changes.

For more on keys and modes, see \see{modes}.
\end{document}
