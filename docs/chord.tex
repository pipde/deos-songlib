\documentclass{article}  
\input{header}

\htmltitle{Songlib: chord}

\title{Songlib: chord\\
\date{Revision Date: \today}}

\author{Song Li Buser}

\date{Revision Date: \today}

\begin{document}

\maketitle

\W\subsubsection*{\xlink{Printable Version}{chord.pdf}}
\W\htmlrule

The chord family of functions plays multiple notes at the same time.
The family follows the {\it n}/{\it r}/{\it d} convention and also has alternatives
where notes are passed in an array rather than variadically.

\begin{verbatim}
    void chord(double duration,int instrument,
        int baseOctave,int basePitch,...,(int) 0);

    void nchord(double duration,int instrument,
        int baseNumberedNote,...,(int) 0);

    void rchord(double duration,RRA *r,...,(int) 0);

    void dchord(double duration,int *data,int length,...,(int) 0);

    void achord(double duration,int instrument,
        int baseOctave,int basePitch,int *offsets,int length);
    
    void nachord(double duration,int instrument,
        int baseNumberedNote,int *offsets,int length);
\end{verbatim}

For {\it chord} and {\it nchord}, the variadic part is a list of
offsets from the numbered base note and is terminated by a zero. For {\it rchord},
the variadic part is additional RRA objects to play simultaneously.
For {\it dchord}, the variadic part is additional data/length pairs.

The {\it achord} function takes a base octave/pitch pair and an array of offsets,
The {\it nachord} function takes a numbered note as the base and an array 
of offsets. The offset arrays have length {\it length}.

As an example, the following line plays a three-note chord:

\begin{verbatim}
    chord(4,guitar,3,C,3,5,7,(int) 0);
\end{verbatim}

The notes played are C, F (which is C + 5 semitones), and G (which is C + 7
semitones). The following calls are equivalent:

\begin{verbatim}
    nChord(4,guitar,C3,5,7,(int) 0);
    rchord(4,getNumberedNote(C3),getNumberedNote(F3),getNumberedNote(G3),(int) 0);
\end{verbatim}

The separation between the starts of the notes in the chord is
controlled by the \seed{stride}{controlFunctions} setting.

\section*{Other simple chord functions}

Certain semitone delta combinations are so common, \songlib\ has shortcut
functions. In the following pairs of calls, the two calls in the pair
are equivalent:

\begin{verbatim}
    chord(H,inst,octave,C,+4,+7,(int) 0);
    maj(H,inst,octave,C);

    chord(H,inst,octave,D,+3,+7,(int) 0);
    min(H,inst,octave,D);

    chord(H,inst,octave,E,+3,+6,(int) 0);
    dim(H,inst,octave,E);

    chord(H,inst,octave,F,+4,+8,(int) 0);
    aug(H,inst,octave,F);
\end{verbatim}

See also: \see{play}, \see{bend},
\see{trill},
\seed{draw/resolve}{draw},
\see{silence},
\see{stride}

\end{document}
