\documentclass{article}  
\input{header}

\htmltitle{Songlib: installation of instrument sample packs}

\title{Songlib: installation of instrument sample packs\\
\date{Revision Date: \today}}

\author{written by: Song Li Buser}

\begin{document}

\maketitle

\W\subsubsection*{\xlink{Printable Version}{install.pdf}}
\W\htmlrule

\section*{Installing the \songlib\ free sample repository}

\Songlib\ is a sample-based music development library. Thus
the quality of the output greatly depends on the quality
of the input samples. The \songlib\ free sample repository
({\it songlib-fsr}) is provided with no restrictions. Because
\seed{rra}{RRA} files are rather large-ish, the samples
are encoded using flac. These flac samples will need to
be converted to {\sc RRA} before \songlib\ can use them. The
basic strategy for unpacking samples is to download
a sample pack to a convenient directory and then (this
is for illustration only; there is a more convenient
method further down):

\begin{verbatim}
    cd convenientDirectory
    tar xvzf instrument-basename.tgz
    ./install
\end{verbatim}

Of course, {\it convenientDirectory} and {\it instrument-basename} are
to be replaced with the appropriate names. Also, you will need to
install {\it flac} utility:

\begin{verbatim}
   sudo apt-get install flac
\end{verbatim}

since the samples are stored as {\it flac} files ({\it flac} is a compressed,
but lossless, audio format). The installation processes converts
the {\it flac} files to {\it rra} format
(hence the need for the {\it flac} utility).
The converted rra note files are then
stored in the {\tt /usr/local/share/samples/}
hierarchy.

The current set of samples in the repository can be found
\xlink{here}{samples}.

There is a Bash shell script, named {\it getpack}, that automates
these tasks. For example, to install the {\it banjo-a.tgz} sample pack,
run the command:

\begin{verbatim}
    getpack banjo-a
\end{verbatim}
    
This will fetch, unpack, and install the instrument sample pack.
Do this in a temporary directory as a bunch of files are left
behind in the process.

\section*{Making your own sample packs}

The converse of {\it getpack} is {\it mkpack}. To make a sample pack,
you start with a set of {\sc WAV} or {\sc RRA} files that are properly named.
Here are three ways to properly name {\sc RRA} files that contain notes
from octave 4, using a filename prefix of {\it soft\_}:

\begin{center}
\begin{tabular}{lll}%
\T\toprule
    {\it note-octave}     &   {\it octave-note}     &   {\it note number}     \\
\T\midrule
\verb!soft_c4.rra!    &   \verb!soft_4c.rra!   &   \verb!soft_48.rra!   \\
\verb!soft_c#4.rra!   &   \verb!soft_4c\#.rra! &   \verb!soft_49.rra!   \\
\verb!soft_db4.rra!   &   \verb!soft_4db.rra!  &   \verb!soft_49.rra!   \\
\verb!soft_d4.rra!    &   \verb!soft_4d.rra!   &   \verb!soft_50.rra!   \\
\verb!soft_d#4.rra!   &   \verb!soft_4d\#.rra! &   \verb!soft_51.rra!   \\
\verb!soft_eb4.rra!   &   \verb!soft_4eb.rra!  &   \verb!soft_51.rra!   \\
\verb!soft_e4.rra!    &   \verb!soft_4e.rra!   &   \verb!soft_52.rra!   \\
\verb!soft_f4.rra!    &   \verb!soft_4f.rra!   &   \verb!soft_53.rra!   \\
\verb!soft_f#4.rra!   &   \verb!soft_4f\#.rra! &   \verb!soft_54.rra!   \\
\verb!soft_gb4.rra!   &   \verb!soft_4gb.rra!  &   \verb!soft_54.rra!   \\
\verb!soft_g4.rra!    &   \verb!soft_4g.rra!   &   \verb!soft_55.rra!   \\
\verb!soft_g#4.rra!   &   \verb!soft_4g\#.rra! &   \verb!soft_56.rra!   \\
\verb!soft_ab4.rra!   &   \verb!soft_4ab.rra!  &   \verb!soft_56.rra!   \\
\verb!soft_a4.rra!    &   \verb!soft_4a.rra!   &   \verb!soft_57.rra!   \\
\verb!soft_a#4.rra!   &   \verb!soft_4a\#.rra! &   \verb!soft_58.rra!   \\
\verb!soft_bb4.rra!   &   \verb!soft_4bb.rra!  &   \verb!soft_58.rra!   \\
\verb!soft_b4.rra!    &   \verb!soft_4b.rra!   &   \verb!soft_59.rra!   \\
\T\bottomrule
\end{tabular}
\end{center}

The last column uses absolute note numbers. The lowest absolute note number
is zero which corresponds to a C in octave 0. The C note in octave one
has a note number of 12, since there are 12 notes in an octave and \songlib\
uses zero-based counting.

Let's suppose I have a set of tinwhistle notes, stored as properly named
{\sc WAV} files, where the {\sc WAV} files start with the prefix {\it soft\_}.
For example, one of the files might be named:

\begin{verbatim}
    soft_f#5.wav
\end{verbatim}

Suppose further, I wish to create a sample pack named {\it tinwhistle-soft}.
I would run this command:

\begin{verbatim}
    mkpack tinwhistle soft soft_*.wav
\end{verbatim}

The first two arguments to {\it mkpack} will be used to construct the sample
pack name.

The {\it mkpack} program will then perform the following steps:

\begin{enumerate}
\item
        convert the {\sc WAV} files to {\sc RRA}
\item
        fade in and fade out the {\sc RRA} amplitudes so that the
        amplitude values both start and end with a zero
\item
        normalize the amplitudes so that all notes have the same volume
\item
        convert the sample notes to FLAC format, which is a lossless, but
        compressed
\end{enumerate}

If {\sc RRA} files are passed to {\it mkpack} instead of {\sc WAV}s, the first step is ommitted.

Next, the FLAC files and an install script are tarred up
into a tarball.
Finally, the tarball is then shipped to the \songlib\ server.
If the notes are already in {\sc RRA} format, then the command would be
something like:

\begin{verbatim}
    mkpack tinwhistle soft soft_*.rra
\end{verbatim}

The fading and normalizing steps will still be performed.
Note: the second argument to {\it mkpack} (in the example, {\it soft})
does not have to match the prefix of the {\sc WAV} or {\sc RRA} files.

For every sample pack,
you should have a file named \verb!<prefix>.README!, where
\verb!<prefix>! is the prefix of your notes (like {\it soft}), that gives
the provenance of the notes in the sample pack. You may
also have a file named \verb!<prefix>.include!
that can be used to simplify the use of your sample pack. Such
an include file might load the sample pack into a predefined
variable, as in:

\begin{verbatim}
    instrument = readScale("/usr/local/share/samples/tinwhistle","soft_");
\end{verbatim}

or for a set of drum kit samples:

\begin{verbatim}
    setCrash(readScale("/usr/local/share/samples/beatbox/","dpe-crash_"));
    setHHOpen(readScale("/usr/local/share/samples/beatbox/","dpe-hhopen_"));
    setHHClosed(readScale("/usr/local/share/samples/beatbox/","dpe-hhclosed_"));
    setHHPedal(readScale("/usr/local/share/samples/beatbox/","dpe-hhpedal_"));
    setSnare(readScale("/usr/local/share/samples/beatbox/","dpe-snare_"));
    setTomHi(readScale("/usr/local/share/samples/beatbox/","dpe-tomhi_"));
    setTom(readScale("/usr/local/share/samples/beatbox/","dpe-tom_"));
    setTomLo(readScale("/usr/local/share/samples/beatbox/","dpe-tomlo_"));
    setKick(readScale("/usr/local/share/samples/beatbox/","dpe-kick_"));
    setRim(readScale("/usr/local/share/samples/beatbox/","dpe-rim_"));
    setStick(readScale("/usr/local/share/samples/beatbox/","dpe-stick_"));
\end{verbatim}


If you don't have an account on the \songlib\ server, this last step will fail.
In this case, you can mail the sample pack to the email address at the
bottom of this document.

\end{document}
