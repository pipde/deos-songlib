\documentclass{article}  
\input{header}

\htmltitle{Songlib: post-processing filters}

\title{Songlib: post-processing filters\\
\date{Revision Date: \today}}

\author{written by: Song Li Buser}


\begin{document}

\maketitle

\W\subsubsection*{\xlink{Printable Version}{post.pdf}}
\W\htmlrule

\section*{Post-processing}

Once an audio file is created with the \songlib\ library, you can enhance
the audio via post-processing. The \songlib\ package comes with a number of
post-processing filters, some of which work natively with RRA files and
some use {\it sox}.
The {\it sox}-based filters can take RRA files as input, but convert
them to WAV before {\it sox} processes them.

\section*{RRA-based filters}

Most RRA filters accept zero, one, or two file names, plus options. If
given no file names, they read from {\it stdin} and write to {\it stdout}. If
given one filename, they read from the given file and write to {\it stdout}.
If given two filenames, they read from the first and write to the second.
Filters marked {\it ABNORMAL} do not follow this pattern.

\subsection*{Format filters}

\begin{description}
\item[rra2mp3,rra2ogg,rra2flac]
batch convert RRA files to MP3, OGG, or FLAC --
this filter requires one or more command-line arguments, all
of which are the names of the RRA files to be converted;
it produces similarly named files with 
\verb!.mp3!, \verb!.ogg!, or \verb!.flac! extensions,
rather than with \verb!.rra! extensions -- {\it ABNORMAL}
\item[flac2rra,flac2mp3,flac2ogg]
batch convert FLAC files to RRA, MP3, or OGG --
this filter requires one or more command-line arguments, all
of which are the names of the FLAC files to be converted;
it produces similarly named files with
\verb!.rra!, \verb!.mp3!, or \verb!.ogg! extensions,
rather than with \verb!.flac! extensions -- {\it ABNORMAL}
\item[ogg2rra,ogg2mp3,ogg2flac]
batch convert OGG files to RRA, MP3, or FLAC --
this filter requires one or more command-line arguments, all
of which are the names of the OGG files to be converted;
it produces similarly named files with
\verb!.rra!, \verb!.mp3!, or \verb!.flac! extensions,
rather than with \verb!.ogg! extensions -- {\it ABNORMAL}
\item[mp32rra,mp32ogg,mp32flac]
batch convert MP3 files to RRA, OGG, or FLAC --
this filter requires one or more command-line arguments, all
of which are the names of the MP3 files to be converted;
it produces similarly named files with
\verb!.rra!, \verb!.ogg!, or \verb!.flac! extensions,
rather than with \verb!.mp3! extensions -- {\it ABNORMAL}
\item[rra2wav,wav2rra]
convert from RRA to WAV and vice versa
\end{description}

\subsection*{Effects filters}

Typically, effects filters are strung together in a pipeline:

\begin{verbatim}
   cat track1.rra | rraamplify -a 0.8 | rrachorus | rrareverb | rrastereo -r > final1.rra
\end{verbatim}

To see the options that a filter handles, use the \verb!-h! option, as in:

\begin{verbatim}
    rraecho -h
\end{verbatim}

\begin{description}
\item[rraamplify]
increase or decrease the volume
\item[rrachorus]
add multiple voices to the incoming audio, some pitched a little bit higher 
and some pitched a little bit lower
\item[rraecho]
add a prominent echo to the incoming audio -- unlike {\it soxecho} (below),
which adds a single echo,
{\it rraecho} re-echos repeatedly,
each subsequent echo fainter than the previous
\item[rrafader]
fade in and fade out the incoming audio
\item[rrafastmixer]
mix a number of audio files, given as command line arguments, into a single
audio file -- the output audio is assumed to be the last filename given as
an argument -- if the last argument is \verb!-!, then the output audio is sent
to stdout.
\item[rraflanger]
add a flanging effect to the incoming audio
\item[rrahiss]
add a hiss to the incoming audio
\item[rramono]
convert a multichannel audio to a single channel -- a single channel from
the input can be selected or all channels in the input can be merged into
a single channel
\item[rrareverb] add some reverb to the incoming audio --
like {\it rraecho}, but the echo is delayed much less
\end{description}

\section*{Mastering filters}

These filters are useful for mastering vocals or instrumentals:

\begin{description}
\item[rraskip]
remove the first part of a song so you can focus on later portions
\item[rrasilence]
reduces the volume on regions that appear to
be silent - it is mainly used to remove
recording hiss and background noises from vocals --
try the default parameters and
adjust if to much or too little signal is being removed
\end{description}

\section*{Sox-based filters}

The {\it sox}-based filters take zero, one, or two command-line arguments. 
If no arguments are given, then the filter expects the input audio to 
come from stdin and the output audio to go to stdout. The default is to
expect a WAV as input and produces a WAV on output:

\begin{verbatim}
    cat song.wav | soxecho > songecho.wav
\end{verbatim}

This example uses the {\it soxecho} filter, but the behavior is the same for
all {\it sox}-based filters.
If the \verb!-r! option is given, then the input on stdin is RRA:

\begin{verbatim}
    cat song.rra | soxecho -r > songecho.wav
\end{verbatim}

If the \verb!-R! option is given, then an RRA is produced on stdout:

\begin{verbatim}
    cat song.wav | soxecho -R > songecho.rra
    cat song.rra | soxecho -r -R > songecho.rra
\end{verbatim}

When one command-line argument is given, it is assumed to be the name
of the audio input file. Sox determines whether it is a WAV or RRA file
by its extension. When two command-line arguments are given, the first
is assumed to be the name of the audio input file and the output is written
to a file named by the second command-line argument. Thus, the following
three commands are equivalent:

\begin{verbatim}
    cat song.rra | soxecho -r -R > songecho.rra
    soxecho -R song.rra > songecho.rra
    soxecho song.rra songecho.rra
\end{verbatim}

Finally, there is the generic {\it soxfilter}, which takes a \verb!-e!
option to
specify the effect and a \verb!-o! option to pass options to the effect.
For example,
the {\it soxecho} filter devolves to the following call to {\it soxfilter}:

\begin{verbatim}
    soxfilter -e echo -o 0.8 0.9 1000.0 0.3
\end{verbatim}

\begin{description}
\item[soxecho]
add a prominent echo to the incoming audio
\item[soxchorus]
add multiple voices to the incoming audio, some pitched a little bit higher 
and some pitched a little bit lower
\item[soxreverb]
add some reverb to the incoming audio -- like {\it soxecho}, but the echo is delayed
much less
\item[soxflanger]
add a flanging effect to the incoming audio
\item[soxbass]
boost the bass of the incoming audio
\item[soxtelephone]
add a telephone effect to the incoming audio -- the low bass and the high treble
are removed to make the audio sound like it is coming over a telephone
\item[soxfilter]
the generic filter on which the above filters are based
\end{description}

\end{document}
