\documentclass{article}  
\input{header}

\htmltitle{songlib: rcombine and runcombine}

\title{songlib: rcombine and runcombine\\
\date{Revision Date: \today}}

\author{Colin B}

\begin{document}

\maketitle

\W\subsubsection*{\xlink{Printable Version}{rcombine.pdf}}
\W\htmlrule

The {\it rcombine} utiltiy is designed to layer multiple RRA files into a single
multi-channeled file.  Multi-channeled input files will have all of their 
layers added, this allows you to rcombine new replicant notes onto ones
that have been previously joined together with {\it rcombine}.
See documentation
on
\seed{randomChannel}{random channel selection} for more information.

\begin{description}

\item[syntax]

\begin{verbatim}
rcombine [-s] inputfile1 inputfile2 ... inputfileN outputfile
\end{verbatim}

\item[options]

The -{\it s}
option activateas silent mode and keeps rcombine from reporting anything back
while it is working.

\item[inputs]

At least 2 input files are required.

\item[output]

The last argument to {\it rcombine} is the file name
to save the new combined RRA file to;
use '-' to have {\it rcombine} print to {\it stdout}.

\end{description}

The {\it runcombine} utiltiy is designed to seperate multi-channeled RRA files 
back into single channeled files.  You can specify a new name for the
output files if you don't want to have it based of the original filename
and can optionally have {\it runcombine} remove the extra data created at the
end of shorter samples when they are joined with longer ones with the
{\it rcombine} utility.

\begin{description}

\item[syntax]

\begin{verbatim}
runcombine [-s -t -n (name)] inputfile1 [inputfile2...inputfileN]
\end{verbatim}

\item[options]

The -{\it s}
option activates silent mode and keeps rcombine from reporting anything back
while it is working.

The -{\it t} option
activates the trim mode which shortens each output file to trim off 
any duplicate data used to pad the file when they were joined together.

The -{\it n} {\it name} option
allows you to set a different name to be used as the basis for the
output file names.

\item[inputs]

The input file names are the file to separate.
You must specify at least one input file. If more than
one file is given, the same basename will be used for all files created 
and the numbering will continue across all tracks.

\item[output]

Each track from the collection of input files will be stored in
a separate RRA file.

\item[examples]

Assuming {\it sample.rra} is a 4 channel RRA file, {\it runcombine} would have this
type of results:

\begin{verbatim}
    runcombine sample.rra   
\end{verbatim}

Files {\it sample00.rra}, {\it sample01.rra}, {\it sample02.rra}, and
{\it sample03.rra}
are generated.

\begin{verbatim}
    runcombine -n note_ sample.rra   
\end{verbatim}
Files {\it note\_00.rra}, {\it note\_01.rra}, {\it note\_02.rra}, and
{\it note\_03.rra}
are generated.

the file numbering matches the original channel numbers that the data
contained in that file was found in!

In the following example, both rra input files have 2 channls:

\begin{verbatim}
    runcombine sampleA.rra sampleB.rra 
\end{verbatim}

Files {\it sampleA00.rra}, {\it sampleA01.rra}, {\it sampleA02.rra}, and
{\it sampleA03.rra}
are generated.

\end{description}

\end{document}
