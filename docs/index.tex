\documentclass{article}  
\input{header}

\htmltitle{Songlib: The C Library for Music Composition}

\title{Songlib: The C Library for Music Composition\\
\date{Revision Date: \today}}

\author{written by: John C. Lusth}

\begin{document}

\maketitle

\W\subsubsection*{\xlink{Printable Version}{index.pdf}}
\W\htmlrule

\section*{Downloading and installing \songlib}

\Songlib\ is distributed in a tarball; the tarball
consists of the songlib source and a set of 
utilities, which also need to be compiled..
The tarball has enough music samples
to run the sample programs in the tarball; to install more
music samples, see \seed{install}{installing sample packs}.

\xlink{Linux and Windows 10 Bash installation instructions}{linux-install.html}\\
\xlink{Mac installation instructions}{mac-install.html}\\
\xlink{Cygwin/Windows installation instructions}{cygwin-install.html}\\

\section{Using \songlib}

\Songlib\ is a simple and (mostly) imperative C
library for creating music. Currently, these documents
exist for explaining how to use {\it\bf songlib}.

\begin{itemize}
\item
    \seed{install}{installing and creating instrument sample packs}
\item
    \seed{songlibQuickStart}{songlib quick start}
\item
    \seed{notePlaying}{the major note-playing functions}
\item
    \seed{readingSamples}{reading note samples}
\item
    \seed{filters}{filtering notes}
\item
    \seed{movements}{moving backwards and forwards in time}
\item
    \seed{controlFunctions}{modifying songlib output}
\item
    \seed{keepingTime}{keeping time}
\item
    \seed{keySignatures}{using key signatures}
\item
    \seed{modes}{modal chords}
\item
    \seed{chords}{special chord generating functions}
\item
    \seed{drums}{special support for drums}
\item
    \seed{noteNotations}{drums and other note notations}
\item
    \seed{voiceLeading}{automatic voice leading}
\item
    \xlink{\Songlib\ without programming - {\it processtab}}{tab.html}
\end{itemize}

Here are the currently available documents on {\it Readily Readable Audio}:

\begin{itemize}
\item
    \seed{rra}{readily readable audio}
\item
    \seed{rraSupport}{reading and writing RRA files}
\item
    \seed{post}{post-processing RRA files}
\end{itemize}

\section*{Recording}

Here is are some short tutorials for laptop recording using
various devices for attaching high quality microphones to
your laptop:

\begin{itemize}
\item
    \xlink{The Blue Icicle}{icicle.html}
\item
    \xlink{The Hosa Tracklink}{hosa.html}
\item
    \xlink{The Mobile-pre}{hosa.html}
\end{itemize}

Here is a link for processing vocals:

\begin{itemize}
\item
    \xlink{Processing vocals}{vocals.html}
\end{itemize}

\section*{Example Music}

Here are some examples of student-made music using \songlib:

\begin{itemize}
\item
        \xlink{Spring 2012}{examples/2012/}
\end{itemize}

\end{document}
