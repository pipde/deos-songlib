\documentclass{article}  
\input{header}

\htmltitle{songlib: play}

\title{Songlib: play\\
\date{Revision Date: \today}}

\author{John C. Lusth}

\begin{document}

\maketitle

\W\subsubsection*{\xlink{Printable Version}{play.pdf}}
\W\htmlrule

The play family of functions is the primary method for playing
samples with {\it\bf songlib}:

\begin{verbatim}
    void play(double duration,int instrument,int octave,int pitch);
    void nplay(double duration,int instrument,int numberedNote);
    void rplay(double duration,RRA * r);
    void dplay(double duration,int *data,int length);
    void splay(double duration,int instrument,int octave,int pitch);
    void snplay(double duration,int instrument,int numberedNote);
    void drum(double duration,int instrument,int numberedNote);
\end{verbatim}

Functions in the
{\it play} family play a given note for the given duration. The duration
is specified in \xlink{beats}{keepingTime.html}.
The note to be played is specified by
an instument/octave/pitch triplet, an instrument/numbered note pair,
an \xlink{RRA}{rra.html} object, or an array of amplitude data.
Note that the play family follows the {n,r,d} convention.

The instrument is an integer returned by the \seed
{readingSamples}{readScale} function.

The actual length of the duration depends on the tempo given
in a \seed{keepingTime}{setTime} call.

For example, if the note is to be played for two beats and
the number of beats per minute is 132 and the sample rate
is 44,100 samples per second, then...

\begin{verbatim}
    2 * 1/132 * 60 * 44100
\end{verbatim}

...or...

\begin{verbatim}
    3272
\end{verbatim}

...samples from the beginning of the given note
will be added into the output buffer starting
at the current location. Moreover, additional
samples beyond the 3727th will be added in,
at ever-diminishing amplitiudes, so that
the note does not end abrubtly.
The number of additional
samples added to the output is governed by
the current value of \seed{controlFunctions}{sustain}.

The exception to the above behavior are the
{\it splay}, {\it snplay} and {\it drum} functions.
These functions play the entire note and then
rewind the song (using 
\seed{movements}{getLocation} and 
\seed{movements}{setLocation})
so that it appears as if only the requested number of
beats has elapsed. The {\it splay} and {\it snplay} functions
are useful for playing instruments whose notes
always have a fixed duration (such as drums).
The function drum randomly adds a little variation
to the amplitude, timing, and pitch of the
note being played.

The {\it play} function and its alternatives
update the current location
in the output.

Here is an example {\it\bf songlib} program that plays a single note:

\htmlrule

\begin{verbatim}
#include "songlib.h"

int
main()
    {
    int inst;

    initSong();
    setTime(132,4);
    setSustain(0.99995);

    openOutput("song.rra",44100,16);

    inst = readScale("../../../samples/guitar","note_");

    play(2.0,inst,3,C);

    closeOutput();

    return 0;
    }
\end{verbatim}

\htmlrule
    
See also:
\see{keepingTime},
\see{controlFunctions},
\see{readingSamples},
\see{movements}
\end{document}
