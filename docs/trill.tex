\documentclass{article}  
\input{header}

\htmltitle{songlib: trill}

\title{Songlib: trill\\
\date{Revision Date: \today}}

\author{John C. Lusth}

\begin{document}

\maketitle

\W\subsubsection*{\xlink{Printable Version}{trill.pdf}}
\W\htmlrule

The {\it trill} family of functions is used to play notes
with trill (a cyclical varying of pitch at the end
of a note). The {\it trill} family follows the \{n,r,d\} convention.

\begin{verbatim}
void trill(double beats,int, instrument,int octave,int pitch,
    double startBeats,double delta,double down,double up,int count);

void ntrill(double beats,int instrument,int numberedNote,
    double startBeats,double delta,double down,double up,int count);

void rtrill(double beats,RRA *r,
    double startBeats,double delta,double down,double up,int count);

void dtrill(double beats,int *data,int length,
    double startBeats,double delta,double down,double up,int count);
\end{verbatim}

The note plays for {\it startBeats} and then oscillates 'count' times
around {\it down} and {\it up}.  Each down phase lasts {\it delta} beats and
each up phase lasts {\it delta} beats.
{\it Down} and {\it up} are specified as offsets.
For example, if the specified note is C3,
{\it down} is -STEP and {\it up} is STEP,
then the note will start at C3 and then oscillate between B2 and Cs3.

A trill is usually considered a rapid alternation between
adjacent notes; the trill function provides more generality
than that.

\end{document}
