\documentclass{article}  
\input{header}

\htmltitle{songlib: creating sample packs}

\title{Songlib: creating sample packs\\
\date{Revision Date: \today}}

\author{written by: Song Li Buser}

\begin{document}

\maketitle

\W\subsubsection*{\xlink{Printable Version}{samples.pdf}}
\W\htmlrule

We start out by assuming you have some WAVE files
representing the notes of an instrument.
Thes files should be named using the following
conventions:

\begin{verbatim}
    <prefix><octave number><note name>.wav
    <prefix><note name><octave number>.wav
    <prefix><midi note number>.wav
\end{verbatim}

For example, the file names

\begin{verbatim}
    note_3c.wav
    note_c3.wav
    note_36.wav
\end{verbatim}

all follow the convention (the prefix is {\tt note\_})
and all represent the same note. In the remainder of this
document, the prefix is assumed to be {\tt note\_}.

In the directory containing the WAVE files, run the command:

\begin{verbatim}
    mkpack <instrument> <prefix> <wav files>
\end{verbatim}

where <instrument> describes the source of the samples and
<prefix> is a prefix of the samples' prefix. For example,
if the notes were generated by a clarinet, the call to
mkpack would be:

\begin{verbatim}
    mkpack clarinet note *.wav
\end{verbatim}

This call will create a sample pack named:

\begin{verbatim}
    clarinet-note.tgz
\end{verbatim}

containing all the given WAVE files
and will attempt to install it on \emph{\bf beastie.cs.ua.edu}. Likely,
you will not have permission to do this, so ignore the
error that is generated.

You should have a file named \verb!<prefix>.README! that gives
the provenance of the notes in the sample pack. You may
also have a file named \verb!<prefix>.include!
that can be used to simplify the use of your sample pack. Such
an include file might load the sample pack into a predefined
variable, as in:

\begin{verbatim}
    instrument = readScale("/usr/local/share/samples/clarinet","note_");
\end{verbatim}

or for a set of drum kit samples:

\begin{verbatim}
    setCrash(readScale("/usr/local/share/samples/beatbox/","dpe-crash_"));
    setHHOpen(readScale("/usr/local/share/samples/beatbox/","dpe-hhopen_"));
    setHHClosed(readScale("/usr/local/share/samples/beatbox/","dpe-hhclosed_"));
    setHHPedal(readScale("/usr/local/share/samples/beatbox/","dpe-hhpedal_"));
    setSnare(readScale("/usr/local/share/samples/beatbox/","dpe-snare_"));
    setTomHi(readScale("/usr/local/share/samples/beatbox/","dpe-tomhi_"));
    setTom(readScale("/usr/local/share/samples/beatbox/","dpe-tom_"));
    setTomLo(readScale("/usr/local/share/samples/beatbox/","dpe-tomlo_"));
    setKick(readScale("/usr/local/share/samples/beatbox/","dpe-kick_"));
    setRim(readScale("/usr/local/share/samples/beatbox/","dpe-rim_"));
    setStick(readScale("/usr/local/share/samples/beatbox/","dpe-stick_"));
\end{verbatim}

Finally, run the command:

\begin{verbatim}
   ./install
\end{verbatim}

to install your new sample pack on your local machine.

\section*{Identifying notes or fixing notes from an out-of-tune instrument}

If you do not know the pitch of a note, you can use the {\it rraidentify}
utility to find the pitch of the note and name it. Suppose the unknown
note is contained in {\it note.rra} and is actually \verb!C#2!.
The command:

\begin{verbatim}
    cp note.rra note_`rraidentify -n note.rra`.rra
\end{verbatim}


should produce the file {\it note\_c\#2.rra},
assuming {\it rraidentify} correctly
identified the pitch.

If you know the pitch of a note, but it came from an instrument that
was out of tune, you can also use rraidentify, in conjunction with
rratune. Suppose the note in a file named {\it note\_c\#2.rra} is 
a little bit sharp. You can repitch the note with the following commands.

\begin{verbatim}
    `rraidentify -p note_c#2.rra`
\end{verbatim}

Note the backticks around the command. Without the backticks,
the command:
    
\begin{verbatim}
    rraidentify -p note_c#2.rra
\end{verbatim}

produces something like:

\begin{verbatim}
    rratune 0.986795 c#2.rra c#2.rra.repitch
\end{verbatim}

The backticks mean ``take the output and run it as a command''. Thus,
after running the command with the backticks, you should find the
file:

\begin{verbatim}
    note_c#2.rra.repitch
\end{verbatim}

has been created. You can then move this 

\begin{verbatim}
    destination=note_`rraidentify -n note.rra`.rra
    cp note.rra $destination
    `rraidentify -p $destination`
    mv $destination.repitch $destination

    `rraidentify -p` > nore_`rraidentify -n
\end{verbatim}

\end{document}
