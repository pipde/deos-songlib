\documentclass{article}  
\input{header}

\newcommand\ver {{1.47}}

\htmltitle{Songlib: The C Library for Music Composition}

\title{Songlib: The C Library for Music Composition\\
\date{Revision Date: \today}}

\author{written by: John C. Lusth}

\begin{document}

\maketitle

\W\subsubsection*{\xlink{Printable Version}{linux-install.pdf}}
\W\htmlrule


\section*{Linux installation instructions}

Begin the installation process by downloading the \songlib\ source
tarball into a directory named {\it songlib}:

\begin{verbatim}
    mkdir ~/songlib
    cd ~/songlib
    wget beastie.cs.ua.edu/songlib/songlib-1.47.tgz
\end{verbatim}

Then, unpack the tarball:

\begin{verbatim}
    tar xvfz *.tgz
\end{verbatim}

If \verb!/usr/local/bin/! is not in your path, then you
should add it. See the web for instructions on how to add
a directory to the \verb!PATH! environment variable.

Install the following supporing packages:

\begin{verbatim}
    sudo apt-get install build-essential sox flac
\end{verbatim}

If you are not on a Debian-based system and don't have {\it apt-get}, 
consult the web on how to install these packages.

For mp3 support, install:

\begin{verbatim}
    sudo apt-get install lame libsox-fmt-mp3 
\end{verbatim}
    
Finally, build and install the library and install the utilities
and sample note files by typing the command:

\begin{verbatim}
    sudo make install
\end{verbatim}

while in the {\it songlib} directory.

To test your installation, move into the {\it quickstart} directory and
type {\it make play}:

\begin{verbatim}
    cd quickstart
    make play
\end{verbatim}

You should see the song being constructed and then should hear it
begin to play.

\end{document}
